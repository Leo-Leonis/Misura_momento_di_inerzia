\begin{center}
\framebox{
\begin{minipage}{5 in}
\begin{center}
\vspace{0.5cm}
\section*{Riassunto}
\end{center}
\textbf{L'esperimento.} Nell'esperimento si è andato a misurare il momento di inerzia di un disco non omogeneo attraverso due metodi: metodo geometrico, analizzando principalmente la geometria del disco, e metodo dinamico, ottenendo la misura attraverso l'utilizzo del pendolo di Maxwell.\\

\noindent \textbf{Risultato.} Nell'esperimento il valore del momento d'inerzia ottenuto attraverso il metodo geometrico è $I_G=(4.91 \pm 0.08) \cdot 10^{-4} \textrm{ kg} \cdot \textrm{m}^2$, mentre per il metodo dinamico è $I_D=(5.1 \pm 0.2) \cdot 10^{-4} \textrm{ kg} \cdot \textrm{m}^2$, per cui si è potuto affermare che i due valori ottenuti sono compatibili entro l'errore.
\vspace{0.5cm}
\end{minipage}}
\end{center}