% pacchetti
\usepackage[utf8]{inputenc} 
\usepackage[italian]{babel}
\usepackage{url}
\usepackage{hyperref}
\usepackage{xcolor}
%\usepackage{nopageno} % per togliere il numero delle pagine
\usepackage{graphicx} % per le immagini
\usepackage{amsthm}
\usepackage{amsmath}
\usepackage{amssymb}
\usepackage{wrapfig} % per le immagini incorniciate con il testo
\usepackage{fancyhdr} % per i headings e footers
\usepackage[backend=biber, style=numeric, defernumbers, sorting=none]{biblatex} % per la bibliografia e la sitografia
\usepackage{eurosym} % per il simbolo dell'euro "€" più carino
\usepackage{appendix}
\usepackage{geometry}
\usepackage{extpfeil}
\usepackage{tikz}
\usepackage{multirow}


% \title{Calcolo differenziale e applicazioni alla fisica\\
% \\\vspace{8pt} {\large \sc su come funzionano le pale eoliche}\\
% \vspace{32pt} {\Large Elaborato per l'Esame di Stato}\\
% \vspace{16pt}}
% \author{\sc Leandro Ye}
% \date{31 maggio 2021}

% impostazioni del pacchetto fancyhdr
\pagestyle{fancy}
\fancyhf{}
\rhead{\sc leandro ye}
\lhead{Misura del momento di inerzia di un disco}
\fancyfoot[c]{\thepage}

% impostazioni del pacchetto eurosym
\DeclareUnicodeCharacter{20AC}{\euro}

% per indicare le fonti delle citazioni
\addbibresource{citazioni.bib}

