%% 2.1 Acquisizioni dati
Qui vengono riportati i valori misurati direttamente per i due metodi.

\subsubsection*{Metodo geometrico}
Per le misure del metodo geometrico è stato fornito direttamente il valore della densità $\rho$ della plastica di cui sono composti i toroidi del disco, e ciò equivale a
$$\rho = (1.36 \pm 0.02) \cdot 10^3 \textrm{ kg/m} ^3$$
Mentre i valori dei diametri $d_i$, delle profondità $p_i$  e di $z_4$ sono riportati in Tab. \ref{diametri_e_spessori}.\\

    \begin{table}[htp]
        \centering
        \begin{tabular}{||c|c|c|c|c||c|c|c||c||}
            \hline \hline
            \multicolumn{9}{||c||}{Misure delle lunghezze per il metodo geometrico (in mm)} \\
            \hline \hline
            $d_1$ & $d_2$ & $d_3$ & $d_4$ & $d_5$ & $p_1$ & $p_2$ & $p_3$ & $z_4$ \\
            \hline
            7.86 & 15.95 & 57.99 & 99.88 & 120.05 & 8.13 & 3.06 & 13.11 & 29.96 \\
            7.82 & 15.93 & 57.99 & 99.89 & 120.03 & 8.17 & 3.06 & 13.16 & 29.95 \\
            7.78 & 15.93 & 57.97 & 99.87 & 120.05 & 7.93 & 3.06 & 13.09 & 29.96 \\
            7.95 &       &       &       &        & 8.17 & 3.16 & 13.19 & 29.95 \\
            7.88 &       &       &       &        &      &      &       &       \\
            \hline \hline 
        \end{tabular}
        \caption[\small Misure delle lunghezze per il metodo geometrico.]{\small I valori riportati in mm delle misure effettuate sui toroidi, in particolare i diametri $d_i$, le profondità $p_i$, e lo spessore del toroide più esterno $z_4$. Da notare come le misure di $d_1$ fluttuino di più rispetto alle altre misure, proprio perché è stato difficile prendere la misura in corrispondenza del diametro più interno.}
        \label{diametri_e_spessori}
    \end{table}

\subsubsection*{Metodo dinamico}
Le 40 misure del tempo vengono riportate su Tab. \ref{tab_tempi_di_caduta}.
    \begin{table}[htp]
        \centering
        \begin{tabular}{||cccccccc||}
            \hline \hline
            \multicolumn{8}{||c||}{Misure del tempo di caduta del disco (in s)} \\
            \hline \hline
            7.31 & 7.41 & 7.39 & 7.49 & 7.32 & 7.56 & 7.42 & 7.69 \\
            7.52 & 7.52 & 7.32 & 7.47 & 7.56 & 7.54 & 7.34 & 7.56 \\
            7.37 & 7.51 & 7.34 & 7.36 & 7.29 & 7.32 & 7.46 & 7.39 \\
            7.39 & 7.37 & 7.39 & 7.24 & 7.29 & 7.57 & 7.41 & 7.37 \\
            7.44 & 7.44 & 7.41 & 7.46 & 7.36 & 7.42 & 7.54 & 7.44 \\
            \hline \hline 
        \end{tabular}
        \caption[\small Misure dei tempi di caduta.]{\small Nella tabella sono riportate i 40 valori del tempo di caduta del disco, in secondi, misurate con il cronometro.}
        \label{tab_tempi_di_caduta}
    \end{table}
Mentre le misure del diametro del filo $d_f$ e del perno metallico $d_p$ e dell'altezza di caduta $h_o$ sono riportate in Tab. \ref{altre_misure}.

    \begin{table}[htp]
        \centering
        \begin{tabular}{||c|c||c||}
            \hline \hline
            $d_f$ [mm] & $d_p$ [mm] & $h_o$ [cm] \\
            \hline
            0.45 & 3.00 & 35.4 \\
            0.40 & 3.02 & 35.4 \\
            0.41 & 3.02 & 35.6 \\
            \hline \hline 
        \end{tabular}
        \caption[\small Misure delle lunghezze per il metodo dinamico.]{\small Le misure del diametro del filo $d_f$ e del perno $d_p$ (in mm), dell'altezza di caduta del disco $h_o$ (in cm).}
        \label{altre_misure}
    \end{table}
\newpage