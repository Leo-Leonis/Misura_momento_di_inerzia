% Introduzione
In questo esperimento si è cercato di misurare il momento di inerzia di un oggetto: ciò si basa sul fatto che il momento d'inerzia può essere ricavato sia dalla geometria dell'oggetto e sia dal suo comportamento nel moto traslarotatorio.
Ogni corpo possiede un fattore di resistenza ad moto rotatorio attorno ad un asse, fattore che in questo caso corrisponde al suo \textbf{momento d'inerzia}.\\
Per esempio un punto materiale di massa $m$ che ruota attorno ad punto distante $r$ ha momento d'inerzia 
$$I = m r^2$$
Oppure, considerato un corpo rigido cilindrico che ruota attorno alla sua asse, con raggio $r$ e massa $m$, si può dimostrare$^{\textrm{\cite{meccanica}}}$ che possiede
$$I =  \frac{1}{2} m r^2$$
come momento d'inerzia.\\
L'aspetto fondamentale che ci permette di ricavare il valore del momento di inerzia è la sua \textbf{additività}: se un corpo rigido è composto da $n$ parti, il momento d'inerzia totale del corpo  è dato dalla \textit{somma dei momenti d'inerzia delle $n$ parti} che costituiscono il corpo, calcolati tutti rispetto allo stesso asse di rotazione.\\
Nell'esperimento il corpo in questione è un disco non omogeneo composto da quattro toroidi concentrici (Fig. \ref{disco}), quindi il momento d'inerzia ricavato mediante procedura geometrica equivale alla somma dei momenti d'inerzia dei quattro toroidi.\\

Alternativamente, è possibile ricavare il suo momento d'inerzia a seconda di come si muove dinamicamente il corpo. Infatti, attraverso l'apparato del Pendolo di Maxwell, il disco assume inizialmente una posizione ad una altezza stabilita, e una volta lasciata libera questo comincia a calare di quota assumendo un moto rototraslatorio, quindi acquistando nello stesso tempo sia energia cinetica traslatoria e sia quella rotazionale, fino ad arrivare a quota zero. Dal momento che \textit{l'energia cinetica rotazionale di un corpo rigido è direttamente proporzionale al suo momento d'inerzia}, si deduce che è possibile ricavare il valore a partire dalla \textbf{legge di conservazione dell'energia meccanica}. Infatti, un corpo rigido ha energia cinetica rotazionale $K_r$ corrispondente a
$$\displaystyle K_r = \frac{1}{2} I \omega ^2$$
con $I$ il momento d'inerzia del corpo e $\omega$ la velocità angolare rispetto all'asse di rotazione.
